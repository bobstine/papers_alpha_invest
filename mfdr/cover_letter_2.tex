\documentclass[12pt]{letter}

\usepackage[american]{babel}
\renewcommand{\baselinestretch}{1.2}   % double spaces a document
\usepackage[usenames]{color}
\newcommand{\dpf}[1]{\noindent{\textcolor{Blue}{\{{\bf dpf:} \em #1\}}}}
 
\setlength{\topmargin}{0in}         % 1 in top/bottom
\setlength{\headheight}{0.25in}
\setlength{\headsep}{0.25in}
\setlength{\textheight}{8.0in}
\setlength{\oddsidemargin}{0.0in}   % 1. in left/right
\setlength{\textwidth}{6.5in}
\pagestyle{empty}  % so no page numbers

\begin{document}

\noindent
October 5, 2006

\vspace{0.3in}

\noindent
Professor A. T. A. Wood\\
School of Mathematical Sciences \\
University of Nottingham \\
University Park \\
Nottingham NG7 2RD \\  
United Kingdom

\vspace{0.3in}

\noindent
Dear Professor Wood: 

\noindent
Please find enclosed our revision to (Please note the new title):

\noindent
{\bf JRSSB submission B6215}: ``Alpha investing: A new multiple
hypothesis testing procedure that controls mFDR.'' (with Foster)

\noindent
We have substantially rewritten the entire paper to address the
suggestions and comments.  We have removed the EDC criterion which we
introduced in the prior version and
instead concentrate on an existing criterion, namely mFDR.  This puts
the focus on the new procedure, namely alpha investing.  We have
added several new theorems to help show how the new procedure differs
from existing procedures.

On our own rereading, we agreed that the original version was too
terse.  Hopefully this version will be easier on the referees and more
attractive to them.

We enclose a detailed response to all the comments we received,
starting with your comments, the AE's, and then those of the two referees.

\vspace{0.3in}
\noindent
Sincerely,
\vspace{0.3in}

\noindent
Robert A. Stine  \\
stine@wharton.upenn.edu



\newpage
\centerline{\bf Wood's comments}

Paragraph 1: We have deleted the EDC criterion which was introduced
in the previous version and replaced it by
 a slight modification of the mFDR criterion.  

We have rewritten the entire paper to simplify it and clarify the
presentation. 

Paragraph 2: Our first choice is still JRSS!  Since so much of the
 orginal research of this nature was published there, we would prefer to
 have our paper in JRSS also.  Hopefully this version is more
 deserving.



\newpage
\centerline{\bf AE's report}

paragraph 1: We have taken to heart the criticism that introducing EDC
was unnecessary and have removed it.  In its stead, we use mFDR
(or more accurately, a slight modification of mFDR).  We have a
section relating our version of mFDR to several of the other versions
of FDR (cFDR, pFDR, eFDR).

We honestly don't see much of a connection to the tail probabilities
of the false discovery proportions.  But we made an effort to generate
a a result (without proof) that is similar in spirit to the tail
probability results.  (See statements 4 and 5 after theorem 3.)  We
left out the proof since it is standard for martingales.

paragraph 2: We don't need EDC to handle the sequential nature for
evaluation.  But we do need the new alpha investing rule that is
presented in the paper.  So this focus has been removed.



\newpage
\centerline{\bf Response to referee \#1 titled: ``Report on B6215''}

paragraph 1 \& 2 \& 3: Agreed!

paragraph 4: We now compare to a variety of existing criteria.  We
felt the dependent testing was both difficult and not relevant
for the ideas we discuss and have removed that section.

paragraph {\bf The criterion.} We removed the concept of EDC and
 replaced it by a modification of mFDR.  We show that controlling our
 stopped version of mFDR makes almost all FDR-like definitions
 equivalent.

GW02, GW04 paragraph: Thanks for pointing out this connection.  We
have added a discussion of these papers in the ``Relating mFDR to
FDR'' subsection.  

Choice paragraph: We removed the EDC criterion and use mFDR. 
We have worked with that criterion throughout.  Interesting side light: 
Yoav Benjamini said that he actually considered mFDR first and then
switched his focus to FDR.

{\bf Dependent tests.} The issue of dependent testing gets very
difficult.  We have removed the extensive discussion of it since we
didn't feel it was laid out clearly enough.  We are working on
developing an algorithm that will generate $\alpha_m$ hypothesis tests
without having to orthogonalize the tests.  To make this
precise, we need some new theorems about Gibbs sampling of convex sets
before we can claim to have a complete solution. (We need a slight
extension of the recent work of Santosh Vempala, ``Sampling
Integration and Optimization of High-dimensional Log-concave
Functions.''  We have been pushing Santosh to include these extensions
in a review paper he is writing.  So we will wait until that is out
before tackling our application.)  In our more applied research, we
have tended to simply ignore the dependence in the tests and cross our
fingers.  It empirically works well in data mining.  We hope to pursue
that in yet a different paper.

{\bf Minor remarks.} 

a) New graphs, hopefully to your liking.

b) Fixed.

{\bf References}: Both have been added.



\newpage
\centerline{\bf Response to referee \#2 titled: ``B6215-ref2''}

{\bf Your opening paragraph:} We apologize for being confusing in the
 first version.  Hopefully this version will be easier to track and
 your initial enthusiasm will return.

{\bf ``(1)''} The introduction has been significantly rewritten.
Hopefully it is clearer now. 

{\bf ``(2)''} We have removed the EDC and replaced it by the slightly modified version of mFDR.
This helps focus attention on the sequential aspect (which you
correctly identified as our primary contribution) and the new
algorithm, alpha investing.  Though the ratio of expectations has some
undesirable properties, we show that in our setting it is often close
to the other more traditional criteria.  The numerical equivalence of
5 different criteria is one of the new theorems in this version.

{\bf ``(3)''} We have thought hard about the language to use here.  We
want to leverage the idea that our approach is a modification of alpha
spending rules.  The word ``spending'' correctly suggests that we are
burning through alpha and never get any new alpha back.  So we want a
metaphor that suggests that we are spending alpha, but that we can
earn new alpha to spend.  So the ``investment'' language is best
thought of as a metaphor, but we give it no more meaning than the
spending analogy.  So neither alpha spending nor alpha investing
advocate that alpha-wealth is appropriate as a  utility function.

{\bf ``(4)''} We have tried to make it clear that an alpha investing
rule is representative of a class of rules.  Some of these rules might make
multiple passes over the hypothesis, others won't.  Some might only
test further hypothesis when the primary endpoints have been proven
(i.e. rejected).  Others might save alpha for the future.  All of
these are design decisions that go into creating the testing rule.  To
help clarify this, we have added a new section called ``designing
alpha investing rules.''

{\bf ``In summary:''} We have rewritten the entire paper.  We have
added several new theorems that will hopefully show what it is that we
have accomplished.  We have added a discussion of how to design rules.

{\bf Detailed comments:} We have made all the changes you suggested.

\newpage
{\bf Changes made in new revision of B6215: Foster and Stine}

A quick ``diff'' showed 1800 changes in the latex source file.  So we
 will not provide a line by line list of changes.  But  major
 changes are listed below:

\begin{description}
\item[EDC $\to$ mFDR] The most visible change, is removing the new
criteria of EDC.  We have replaced it with mFDR.  This allows a more
direct comparison to other versions of FDR.
\item[Language] Much of the language has been changed to better match
standard usage.
\item[Stopped testing] We have some new results concerning stopping
the testing.  In particular, the new theorem 1 converts our results to
a ``conditional'' form of FDR.  These ideas are further elaborated in
the new section entitled ``Universal mFDR.''
\item[Stopping time discussion] We discuss a situation where one can
expect to be able to stop at a fixed number of rejections.  This
allows a very close connection between 5 versions of FDR.
\item[Figures] New figures have been generated to match the use of mFDR.
\item[Removed example] We removed the dependent testing example from
the {\em Examples} section.  We feel that to do it justice will take
a paper in its own right.  So we defer it to another day.
\item[Designing alpha investing rules] A new section has been added to
help understand issues that go into designing alpha investing rules.
This will also help set up the rules that we discuss in the examples
section. 
\end{description}


\end{document}  
