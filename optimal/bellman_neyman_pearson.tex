% -*- mode: LaTex; outline-regexp: "\\\\section\\|\\\\subsection";fill-column: 80; -*-
\documentclass[12pt]{article}
\usepackage[longnamesfirst]{natbib}
\usepackage[usenames]{color}
\usepackage{graphicx}  % Macintosh pdf files for figures
\usepackage{amssymb}   % Real number symbol {\Bbb R}
\input{../../standard}

% --- margins
\usepackage{../sty/simplemargins}
\setleftmargin{1in}   % 1 inch is NSF legal minimum
\setrightmargin{1in}  % 1 inch is NSF legal minimum
\settopmargin{1in}    % 1 inch is NSF legal minimum
\setbottommargin{1in} % 1 inch is NSF legal minimum

% --- Paragraph split, indents
\setlength{\parskip}{0.00in}
\setlength{\parindent}{0in}

% --- Line spacing
\renewcommand{\baselinestretch}{1.3}

% --- Margins
\setlength{\topmargin}{-0.5in}
\setlength{\oddsidemargin}{-0.1in}
\setlength{\textheight}{9.0in}
\setlength{\textwidth}{6.5in}

% --- page numbers
\pagestyle{empty}  % so no page numbers

% --- Hypthenation
\sloppy  % fewer hyphenated
\hyphenation{stan-dard}
\hyphenation{among}

% --- Customized commands, abbreviations
\newcommand{\TIT}{{\it  {\tiny DRAFT (\today)}}}

% --- Header
\pagestyle{myheadings}
\markright{\TIT}

% --- Title

\title{ Bellman Approach to Neyman Pearson }
\author{
        Robert A. Stine                     \\
        Department of Statistics            \\
        \thanks{Research supported by... }  \\
        The Wharton School of the University of Pennsylvania \\
        Philadelphia, PA 19104-6340                          \\
        www-stat.wharton.upenn.edu/$\sim$stine 
}

\date{\today}

%%%%%%%%%%%%%%%%%%%%%%%%%%%%%%%%%%%%%%%%%%%%%%%%%%%%%%%%%%%%%%%%%%%%%%%%%%%

\begin{document}
\maketitle 
%------------------------------------------------------------------------

\abstract{  
}

%------------------------------------------------------------------------
\vspace{0.05in}

\noindent
{\it Key Phrases: } 

\clearpage

% To Do
% 

% ----------------------------------------------------------------------
\section{Introduction}
% ----------------------------------------------------------------------

The Neyman-Pearson lemma is about the solution to the testing problem.

The classical problem sets up a constrained optimization.  Find an
 indictor function (or the associated set) $R_\alpha$ such that
 \begin{displaymath}
  R_\alpha = \sup_{\mbox{\scriptsize indicators } R } \{ \ev_{H_1}(R) 
             \mbox{ where } \ev_{H_0} R \le \alpha \}   
 \end{displaymath}


The Bayes solution inserts a prior and controls the number of
incorrect decisions,
\begin{eqnarray*}
  \mbox{Bayes}(R) 
  &=& \min_R \ev(R|H_0) \pi + \ev(1-R|H_1) (1-\pi) \cr
  &=& \min_R \pi \int_R dF_0 + (1-\pi) - (1-\pi)\int_{R} dF_1 \cr
  &=& (1-\pi) \min_R \int_R \left( \pi dF_0 - (1-\pi) dF_1 \right)
\end{eqnarray*}
The solution is then to set
 \begin{displaymath}
     R: \pi dF_0 < (1-\pi) dF_1   
 \end{displaymath}
or 
 \begin{displaymath}
    \frac{dF_0}{dF_1} < \frac{1-\pi}{\pi}   
 \end{displaymath}


The connection to the constrained Bellman optimization approach is
 seen in a graph with x-axis $\ev(R|H_0)$ (which defines the
 alpha-level) and y-axis $\ev(1-R|H_1)$.  The prior weight $\pi$
 'pushing' the convex function inward toward the origin as it defines
 a negatively sloping tangent line.
%--------------------------------------------------------------------------
\section*{Acknowledgement}
%--------------------------------------------------------------------------


%--------------------------------------------------------------------------
% References
%--------------------------------------------------------------------------

\bibliography{../../../references/stat}
\bibliographystyle{../bst/asa}

\end{document} %==========================================================
