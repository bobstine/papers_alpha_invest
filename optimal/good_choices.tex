\documentclass{article}



\begin{document}


NOTE: This is a .Rnw file since I'm hoping to actually draw some
pictures and maybe even calculate the answer.  But at the moment, it
will complie as pure latex.


\title{Good choices: How to pick the parameters in alpha investing?}

There are three parameters we need to pick:
\begin{itemize}
\item initial wealth ($W_0$)
\item return on investment ($\omega$)
\item Scaling factor on spending rule $k$
\end{itemize}
Define
\begin{displaymath}
T(x) = \sum_{i=x}^\infty t(i)
\end{displaymath}
and
\begin{displaymath}
t(i) = \frac{1}{i \log(i)^2}
\end{displaymath}
Define the bid function:
\begin{displaymath}
B(W) = k t(\lceil T^{-1}(W/k) \rceil))
\end{displaymath}
which says how much to bid if you currently have wealth $W$.  Notice
that $k$ enters twice as a conversion factor: first to convert wealths to
our universal function $T$, and then to convert the output of $t$ back
into probabilities.

<<definitions,echo=FALSE>>=

t <- function(i)
    {
        return(1 / ((i+1) * log(i+1)^2));
    }

# Insert value from mathematica for a more accurate answer
global.T.1 <- (function(x){sum(t(1:x))+1/log(x+1)})(3000000)

global.largest.cache = 10000000
global.T.cumsum.cache = cumsum(t(1:global.largest.cache))

global.omega = .5


T_real <- function(n)
    {
        if(n == 1)
            return( global.T.1 )
        if(n > global.largest.cache)
            return(1/log(n))
        else
            return (global.T.1 - global.T.cumsum.cache[ round(n - 1)])
    }

T <- function(arrayN)
    {
        return( sapply(arrayN,T_real))
    }


T_inverse_bisection <- function(p,n.lower, n.upper)
    {
        if(n.lower >= n.upper - 1)
            return(n.lower)
        if(n.lower > .9999999* n.upper)
            return(n.lower)
        n.middle = floor((n.lower + n.upper)/2)
#        cat(n.middle, T(n.middle),"\n")
        if(T(n.middle) < p)
            return( T_inverse_bisection(p, n.lower, n.middle))
        else
            return(T_inverse_bisection(p, n.middle, n.upper))
    }

T_inverse <- function(p)
    {
       lower.guess = .1  * exp(1/p)
       upper.guess = 10 * exp(1/p)
        return( T_inverse_bisection(p,lower.guess,upper.guess));
    }

@


<<bidPlot,fig=TRUE,echo=FALSE>>=

plot(T(1:1000),t(1:1000),log="xy",xlab="Wealth",ylab="bid amount")

@


\section{Balance: $\omega = .5$}

If a fraction of about $p$ hypothesis are not null, then our emphasis
is on finding signal at about the critical region of $\sqrt{2
\log(1/p)}$.  In other words, if the signal is much higher--it is
trival to find--if it is much lower it is impossible.  So the $\mu$'s
that we are worried about are at this critical value.  Hence, calling
a null significant and estimating it has about the same cost as
missing a significant variable and calling it a zero.  So we would
like to have about the same number of type I errors as we have type II
errors.  This pegs our $\omega$ at .5.

\section{The constant term: $W_0 \approx .5$}

We estimate the constant term automatically.  So if we add a spurious
variable--it won't cause that much damange to the risk we already
substain by having the constant term in the model.  So we want to add
about $O(1)$ new variables.  To pick the actual amount, we can think
of adding the constant term to the regression as a successful
rejection and so we should pay out $\omega$ for it.  Thus our initial
wealth would be $W_0 = \omega$.  but this is just a
heuristic--anything else which is $O(1)$ is find also.

<<firstBid,fig=TRUE,echo=FALSE>>=

ks <- c(.001,.01,.1,.2,.5,1:100,1000,10000)

t_first_bid <- function(k)
   {
        index <- T_inverse(global.omega / k)
        bid <- k * t(index)
        t <- -qnorm(bid)
        return(t)
   }

bids <- sapply(ks, t_first_bid)

plot(ks,bids)


@ 


\section{Scaling: $k = $?}

Ideally, if we had about $p$ fraction of non-null hypothesis, then we
would like to set our threshold at about $p$.  Why?  Because then we
would find as many alternatives as we can but still only add about as
many nulls as we have alternatives.  So again--it is a balance
argument.

So, if we see a sequence if IID hypothesis each with $p$ probability
of being the alternative, then we would like to test all of them at an
alpha of about $p$.

So the actual bids that we will use will be $t(x)$ up to $t(x+1/p)$
where $x$  is defined by:
\begin{displaymath}
\sum_{x=i}^{i+1/p} k t(i) \approx  \omega
\end{displaymath}
We would like these bids to all be about size $p$.  If we are in a
region where $t(i)$ is basically flat (i.e. $x \ne o(1/p)$, or in
words, $x$ is about as large is $1/p$ or larger).  So we want
\begin{eqnarray*}
\sum_{x=i}^{i+1/p} k t(i) &\approx&  \omega \left(= .5 \right) \\
\frac{1}{p} k t(x) &\approx&  .5 \\
t(x) & \approx& \frac{p}{2k} \\
\frac{1}{x \log(x)^2} & \approx&\frac{p}{2k} \\
k & \approx& \frac{1}{2} p x \log(x)^2
\end{eqnarray*}
We also want $kt(x) \approx p$, so:
\begin{eqnarray*}
 kt(x) &\approx& p \\
\frac{k}{x \log(x)^2} &\approx& p \\
k & \approx&  xp \log(x)^2 \\
\end{eqnarray*}
Since this doesn't solve for $k$, what that means is that for many
$k$'s we can find an $x$ that will work.  So the key fact that we want
then is that
\begin{displaymath}
t(x)/t(x+1/p) \approx 1
\end{displaymath}
This is then where the teeth of this story live.  But this will work
as long as $x \ne o(1/p)$ which just says that $k$ should be large.
So we don't have much to work with here--only making $k$ larger is a
good thing.


<<computingX>>=


recursive_find_x <- function(k,p, lower, upper)
    {
        if(lower >= upper - 1)
            return(lower)
        middle = floor((lower + upper)/2)
        if(k* T(middle) < global.omega + k * T(1+ floor(middle + 1/p)))
            return(recursive_find_x(k,p, lower,middle))
        else
            return( recursive_find_x(k,p, middle,upper))
    }



find_x <- function(k,p)
    {
        if(p > 1)
            cat("oops: find_x(k,p) not the other way around.\n")

        lower = 1
        upper = 10000000
        answer = recursive_find_x(k, p, lower, upper)

        return(answer)
    }

describe <- function(k,p,answer)
{
    if(answer < 1000000)
    {
        cat("find_x: (",k,",",signif(p,2),") --> ",signif(answer,3),"   \tkT(",answer,") = ",k*T(answer)," > ",k*T(answer+1/p)+.5," = kT(",round(answer+1/p),") + .5",sep="")
        cat("\t\t [kt(l)/p,...,kt(u)/p] = [ ",k*t(answer)/p , " ... " , k*t(answer+1/p)/p," ]\n",sep="")
    }
    return(answer)
}

@

The following output should be interpreted as follows.  The result of
find\_x is the point where and IID sequence with probably $p$ of the
alternative will jump up to when it gets a hit.  For example, if
$p=.01$, then about every 100 rounds we will get a hit.  From
find\_x(.01) = 19, we see that the process will spend down to about
kT(119) and then get a hit and jump back up to about 19.  This make
sense since kT(119) is about .83 (and if it jumps later, it will still
be about .82.)  So when it jumps back up, it will jump up to a value
of about 1.3 which means the process will be at a kT wealth of about
19.  When it starts out at 19 it is spending at a rate much faster
than .01, and when it finishes and finally gets a hit it is spending
at a rate much slower.  In fact it is starting out spending at 2.2
times the target rate and end up .14 times the target rate.

<<tableOfValues>>=

table <- sapply(1:100, function(x){value=signif(.8^x,0); c(value,describe(4,value,find_x(4,value)))} )


@



But we don't want $k$ to be too large.  So we need a more refined
argument.

What we want to happen is to have a rejection region that for IID with
$p$ fraction alternatives looks like a rejection region which has a
alpha of $p$.  But we know that for very small $p$, this is not very
sensitive to $p$.  So if we had a cut off of $z_a$ when we would like
to have used $z_b$, if $|z_a - z_b|$ is small, we don't really care.
But the $P(Z > z_a)/P(Z > z_b)$ could be huge.  So let's pick an
arbitary tuning parameter, call it $c$.  We want to have $|z_a-z_b|<c$
for our IID sequence.  This means that we would like to have
\begin{displaymath}
|\Phi^{-1}(k t(x+1/p)) - \Phi^{-1}(k t(x+1/p))| \le c
\end{displaymath}
for the appropiate $x$.  We couldn't achieve this for any finite $k$
since there would always be a $p$ sooooo small that when we get a hit
it jumps to a much larger bid amount than it should.  But this isn't
really a problem since that leads to over fitting.  But we know we
don't over fit.  So the real problem is if the
\begin{displaymath}
\Phi^{-1}(k t(x+1/p)) > \Phi^{-1}(p) + c
\end{displaymath}
Then we will start missing signal that we care about finding.  This
now isn't a problem for extremely small $p$ since then $x \approx
o(1/p)$ and so we want:
\begin{displaymath}
\Phi^{-1}(k p /\log(1/p)^2) < \Phi^{-1}(p) + c
\end{displaymath}
which will be true if $p$ is small enough.  So it is only checking for
large values of $p$ that we care about.  So we now have an
optimazation to solve:
\begin{itemize}
\item Pick the smallest $c$ such that
\item there is a $k$ such that
\item for all $p$, if we define $x$ so that
\begin{displaymath}
k  \approx  xp \log(x)^2
\end{displaymath}
\item then
\begin{displaymath}
\Phi^{-1}(k t(x + 1/p)) < \Phi^{-1}(p) + c
\end{displaymath}
\end{itemize}

My guess is that $k \approx 4$ is about right.  Why?  I like it.

\section{And the calculation}

Ok, I'm going to try it slightly differently.  For each value of $k$
I'll compute the maximum
\begin{displaymath}
\Phi^{-1}(k p /\log(1/p)^2) - \Phi^{-1}(p)
\end{displaymath}

<<calculation,echo=FALSE>>=

max_gap <- function(k,ps)
  {
    result <- 0
    location  <- 0
    for(i in 1:length(ps))
      {
        p <- ps[i]
        x <- find_x(k,p)
        final.bid <- k * t(x + floor(1/p))
        final.t <- -qnorm(final.bid)
        target.t <- -qnorm(p)
        diff <- final.t - target.t
        if(diff > result)
          {
            result <- diff
            location <- i
          }
      }
    cat(k,ps[i],result,"\n")
    return( result )
  }

ps <- .5 * .9 ^ (1:500)

 sapply(c(.01,.1,.5,1:10,20,30,100,1000,100000), function(x){max_gap(x,ps)})

@ 

\end{document}
