% -*- mode: LaTex; outline-regexp: "\\\\section\\|\\\\subsection";fill-column: 80; -*-
\documentclass[12pt]{article}
\usepackage[longnamesfirst]{natbib}
\usepackage[usenames]{color}
\usepackage{graphicx}  % Macintosh pdf files for figures
\usepackage{amssymb}   % Real number symbol {\Bbb R}
\input{../../standard}

% --- margins
\usepackage{../sty/simplemargins}
\setleftmargin{1in}   % 1 inch is NSF legal minimum
\setrightmargin{1in}  % 1 inch is NSF legal minimum
\settopmargin{1in}    % 1 inch is NSF legal minimum
\setbottommargin{1in} % 1 inch is NSF legal minimum

% --- Paragraph split, indents
\setlength{\parskip}{0.1in}
\setlength{\parindent}{0in}

% --- Line spacing
\renewcommand{\baselinestretch}{1.3}

% --- Margins
\setlength{\topmargin}{-0.5in}
\setlength{\oddsidemargin}{-0.1in}
\setlength{\textheight}{9.0in}
\setlength{\textwidth}{6.5in}

% --- page numbers
\pagestyle{empty}  % so no page numbers

% --- Hypthenation
\sloppy  % fewer hyphenated
\hyphenation{stan-dard}
\hyphenation{among}

% --- Customized commands, abbreviations
\newcommand{\TIT}{{\it  {\tiny Martingale explained (\today)}}}

% --- Header
\pagestyle{myheadings}
\markright{\TIT}

% --- Title

\title{ Explaining the Martingale in Alpha-Investing }
\author{
        Robert A. Stine                     \\
        Department of Statistics            \\
        The Wharton School of the University of Pennsylvania \\
        Philadelphia, PA 19104-6340                          \\
        www-stat.wharton.upenn.edu/$\sim$stine 
}

\date{\today}

%%%%%%%%%%%%%%%%%%%%%%%%%%%%%%%%%%%%%%%%%%%%%%%%%%%%%%%%%%%%%%%%%%%%%%%%%%%

\begin{document}
\maketitle 
%------------------------------------------------------------------------

\abstract{  

This note explains the origins of the martingale used in the JRSS paper and how
it can be ``simplified'' as worked out in the correlated test manuscript.

}

% ----------------------------------------------------------------------
\section*{ JRSS version }
% ----------------------------------------------------------------------

 The martingale of interest in the paper is motivated by showing that the mFDR is
 bounded above by $\alpha$,
 \begin{equation}
   \frac{\ev V(j)}{\ev R(j)+\eta} \le \alpha \;,    
 \label{eq:bound}
 \end{equation}
 where $V(j)$ is the cumulative number of incorrectly rejected hypotheses,
 $R(j)$ is the cumulative number of rejected hypotheses, and $\eta > 0$ is a
 constant used to control the bound when the complete null holds.  It does not
 appear in the martingale.  The bound \eqn{eq:bound} holds if
 \begin{equation}
    0 < \alpha(\ev R(j)+\eta) - \ev V(j)   
 \label{eq:bound2}
 \end{equation}
 We can prove this result if we can show that the expected value of increments
 that go into the cumulative sum are non-negative. (Since we study the
 increments, $\eta$ drops out.) In the end, the martingale that we study is one
 that looks harder to work with since it subtracts a further positive term, the
 cumulative wealth $W(j)$ of the process,
 \begin{displaymath}
      A(j) = \alpha(\ev R(j)+\eta) - \ev V(j) - W(j) \;.
 \end{displaymath}
 It looks rather puzzling that it is easier to show that this inequality holds,
 even though we have subtracted a further positive term.


 The sum in \eqn{eq:bound2} has increment
 \begin{equation}
   \alpha\,R_j - V_j    
 \label{eq:increment}
 \end{equation}
 Consider this term first if the tested null hypothesis $H_j$ is false, then
 when $H_j$ is true.  The latter is the harder situation.  If $H_j$ is false,
 then we cannot falsely reject the null hypothesis, so
 \begin{displaymath}
     H_j \mbox{ is false } \quad \Rightarrow \quad V_j \equiv 0
 \end{displaymath}
 The increment is then
 \begin{displaymath}
     H_j \mbox{ false: }  \alpha\,R_j \ge 0 \mbox{ a.s.}
 \end{displaymath}
 If however the null is true, then $V_j \equiv R_j$, the level of the test is
 $\ev R_j = \alpha_j$, and the expected value of the increment is negative:
 \begin{displaymath}
     H_j \mbox{ true: }  
      \ev(\alpha\,R_j - V_j) = \ev(\alpha\,R_j - R_j) = \alpha_j(\alpha-1) < 0.
 \end{displaymath}


 The martingale that works comes from ``giving up'' some of the fat cushion we
 have when $H_j$ is false to compensate for what happens when $H_j$ is true.
  For the compensator, we use the change in the wealth of the process.  The
 wealth typically goes up when $H_j$ is false and goes down on average when
 $H_j$ is true, so we subtract this from the increment.  Its a bit like an
 antithetic variable in a simulation.  The increment is then ($\omega \le
 \alpha$ is the payout when a hypothesis is rejected.)
 \begin{eqnarray*}
   \alpha\,R_j - V_j - W_j 
    &=&  \alpha\,R_j - V_j - [\omega R_j - (1-R_j)\alpha_j] \cr
    &=& (\alpha-\omega)\,R_j - V_j + (1-R_j)\alpha_j  \cr
    &\ge& (1-R_j)\alpha_j - V_j 
 \label{eq:increment2}
 \end{eqnarray*}
 We have so much cushion that we can give up a bit right there at the
 start. (Since $\omega \le \alpha$, we might not have had this part anyway!)
 This increment is again positive if $H_j$ is false,
 \begin{displaymath}
    H_j \mbox{ false: }  (1-R_j)\alpha_j - V_j \equiv (1-R_j)\alpha_j \ge 0
 \end{displaymath}
 If $H_j$ holds,
 \begin{displaymath}
    H_j \mbox{ true: }  (1-R_j)\alpha_j - V_j \equiv (1-R_j)\alpha_j - R_j
 \end{displaymath}
 In expectation, the increment is (p-values are uniform when $H_j$ holds)
 \begin{displaymath}
  \ev \left( (1-R_j)\alpha_j - R_j \right) 
    = (1-\alpha_j)\alpha_j - \alpha_j 
    = \alpha_j^2 - \alpha_j < 0
 \end{displaymath}
 That's not quite enough, so we increased the cost for bidding to
 $\alpha_j/(1-\alpha_j)$ and the expected value becomes
 \begin{displaymath}
   \ev \left( (1-R_j)\frac{\alpha_j}{1-\alpha_j} - R_j \right)
    = \alpha_j - \alpha_j = 0 
 \end{displaymath}


%--------------------------------------------------------------------------
\section*{ Simplified Version in Multiple Endpoints }
%--------------------------------------------------------------------------

 This version of the martingale removes the tuning parameters $\omega$ and
 $\eta$ and simplifies the wealth process.  Set the payout to the initial wealth
 $\omega = \alpha$ and set the offset in the denominator $\eta = 1$.  The wealth
 process in the JRSS version increases by $\omega$ when rejecting $H_0$ and pays
 $\alpha_j/(1-\alpha_j)$ when it does not reject.  Notice that it does not
 ``pay'' the bid amount if the hypothesis is rejected.  The version used in this
 paper operates differently in that it {\em always} pays the bid $\alpha_j$.
  (Note: The C++ implementation works this way; it is much simpler to program if
 you simply collect the winning bid from the get go.)


 These changes simplify the increment used in the JRSS paper from
 \begin{displaymath}
   \omega R_j - (1-R_j)\alpha_j/(1-\alpha_j)
 \end{displaymath}
 to
 \begin{displaymath}
  \alpha\, R_j - \alpha_j \;.   
 \end{displaymath}
 That is, the expert always pay its bid $\alpha_j$ and earns $\alpha$ if the
 hypothesis is rejected.  The increment in the martingale is then (compare to
 \eqn{eq:increment2})
 \begin{eqnarray*}
   \alpha\,R_j - V_j - W_j 
    &=&  \alpha\,R_j - V_j - [\alpha R_j - \alpha_j] \cr
    &=& \alpha_j - V_j  
 \label{eq:increment3}
 \end{eqnarray*}
 Obviously, the increment of the process $A(j)$ is positive when $H_j$ is
 false since then $V_j \equiv 0$.  If the null $H_j$ is true, we again have the
 needed result since then $\ev V_j = \ev R_j = \alpha_j$.


%--------------------------------------------------------------------------
% References
%--------------------------------------------------------------------------

\bibliography{../../../references/stat}
\bibliographystyle{../bst/asa}

\end{document} %==========================================================
